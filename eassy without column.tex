\documentclass[10pt,a4 paper,two coloum] {article}
\title{STORY OF KANGLEIPAK}
\author{MAISNAM AKASH SINGH}
\date{23 October 2021}
\pagenumbering{roman}
\begin{document}
 \maketitle
 \section{INTRODUCTION}
The Manipur Kingdom was an ancient independent kingdom at the India–Burma frontier that was in subsidiary alliance with British India from 1824, and became a princely state in 1891. It bordered Assam Province in the west and British Burma in the east, and in the 20th century covered an area of 22,327 square kilometres (8,621 sq mi) and contained 467 villages. The capital of the state was Imphal.The early history of Manipur is composed of mythical narratives. The Kangla Fort, located on the banks of the Imphal River, is where the palace of King Pakhangba was located. It was built in 1632 by king Khagemba, who had defeated Chinese invaders. In the fort, a number of temples that had traditional religious significance are located. Kangla means "dry land" in the old Meitei language.
\section{KANGLEIPAK STATE}
The Kingdom of Kangleipak was established by King Loiyumba in 1110 who ruled between 1074 and 1121. He consolidated the kingdom by incorporating most of the principalities in the surrounding hills and is credited with having enacted a kind of written constitution for his state. After subjugating all the villages within their valley Kangleipak kings grew in power and began a policy of expansion beyond their territory. In 1443 King Ningthoukhomba raided Akla, an area ruled by Shan people, initiating a policy of Manipuri claims to the neighbouring Kabaw Valley.The zenith of the Kangleipak State was reached under the rule of King Khagemba (1597–1652). Khagemba's brother Prince Shalungba was not happy about Khagemba's rule so he fled to the Sylhet region where he allied with Bengali Muslim leaders. With a contingent of Sylheti soldiers, Shalungba then attempted to invade Manipur but the soldiers were captured and made to work as labourers in Manipur. These soldiers married local Meitei women and adapted to the Meitei language. They introduced hookah to Manipur and founded the Pangal or Manipuri Muslim community.
\section {MANIPUR STATE}
In 1714 a king, named Meidingu Pamheiba, adopted Hinduism as the state religion and changed his name to Garib Niwaj. In 1724 the Sanskrit name Manipur ('the Abode of Jewels') was adopted as the name of the state. King Garib Niwaj made several incursions into Burma, but no permanent conquest. After the death of Gharib Nawaz in 1754, Manipur was occupied by the Kingdom of Burma and the Meitei king Jai Singh (Ching-Thang Khomba) sought help from the British. A treaty of alliance was negotiated in 1762 and a military force sent to assist Manipur. The force was later recalled and then the state was left to its own devices. Manipur was invaded at the onset of the First Anglo-Burmese War, together with Cachar and Assam.
\section{BRITISH PROTECTORATE}
Following the Burmese invasions, in 1824 the king of Manipur Gambhir Singh (Chinglen Nongdrenkhomba) asked the British for help and the request was granted. Sepoys and artillery were sent and British officers trained a levy of Manipuri troops for the battles that ensued. After the Burmese were expelled, the Kabaw Valley down to the Ningthi River was added to the state.[11] In 1824–1826, on the conclusion of the First Anglo-Burmese War, Manipur became a British protectorate.Manipur remained relatively peaceful and prosperous until King Gambhir Singh's death in 1834. When he died his son was only one year old and his uncle Nara Singh was appointed as regent. That same year the British decided to restore the Kabaw Valley to the Kingdom of Burma, which had never been happy about the loss. A compensation was paid to Raja of Manipur in the form of an annual allowance of Rs 6,370 and a British residency was established in Imphal, the only town of the state, in 1835 to facilitate communication between the British and the rulers of Manipur.After a thwarted attempt on his life, Nara Singh took power and held the throne until his death in 1850. His brother Devendra Singh was given the title of Raja by the British, but he was unpopular. After only three months Chandrakirti Singh invaded Manipur and rose to the throne, while Devendra Singh fled to Cachar. Numerous members of the royal family tried to overthrow Chandrakirti Singh, but none of the rebellions was successful. In 1879, when British Deputy Commissioner G.H. Damant was killed by an Angami Naga party, the king of Manipur assisted the British by sending troops to neighbouring Kohima. Following this service to the crown, Chandrakirti Singh was rewarded with the Order of the Star of India.After Maharaja Chandrakriti's death in 1886 his son Surachandra Singh succeeded him. As in previous occasions, several claimants to the throne tried to overthrow the new king. The first three attempts were defeated, but in 1890, following an attack on the palace by Tikendrajit and Kulachandra Singh, two of the king's brothers, Surachandra Singh announced his intention to abdicate and left Manipur for Cachar. Kulachandra Singh, the king's younger brother, then rose to the throne while Tikendrajit Singh, an older brother and commander of the Manipuri armed forces, held the real power behind the scenes. Meanwhile, Surachandra Singh, once safely away from Manipur appealed to the British for help to recover the throne.
\section{INCORPORATION INTO INDIA}
On 15 August 1947, with the lapse of paramountcy of the British Crown, Manipur became briefly "independent", i.e., free of control from the Governor of Assam which is considered as reversion to political autonomy that exist before 1891 However, the Maharaja had already agree to accede to India on 11 August, whereby he agreed to cede three subjects namely defence, external affairs and communication to India on assurance of autonomy and independence of Manipur be granted in dealing with the new India (Dominion of India)A 'Manipur State Constitution Act 1947' was enacted, giving the state its own constitution, although this did not become known in other parts of India owing to the relative isolation of the kingdom.The Government of India did not recognize the Constitution.On 21 September 1949, the Maharaja was coerced to sign a Merger Agreement with the Union of India, to take effect on 15 October the same year. As a result of the agreement, the Manipur State merged into the Indian Union as a Part C State (similar to a Chief Commissioner's Province under the colonial regime, later called Union Territory), to be governed by a Chief Commissioner appointed by the Government of India. The representative assembly of Manipur was abolished.Unhappy about central rule, Rishang Keishing began a movement for representative government in Manipur in 1954. The Indian home minister, however, declared that the time was not yet ripe for the creation of representative assemblies in Part C States such as Manipur and Tripura, claiming they were located in strategic border areas of India, that the people were politically backward and that the administration in those states was still weak. However, it was given a substantial measure of local self-government under the Territorial Councils Act of 1956, a legislative body and council of ministers in 1963, and full statehood in 1972.
\end{document}