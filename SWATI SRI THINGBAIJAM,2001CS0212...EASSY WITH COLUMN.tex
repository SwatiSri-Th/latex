\documentclass[16pt, a4 paper,two column]{article}
\title{\textbf{The Milky Way: One of the Many Galaxies}}
\author{SWATI SRI THINGBAIJAM}
\date{23rd October,2021}
\begin{document}
\maketitle
\section{Intoduction}
The idea that each star is a sun, many with their own solar systems, is a powerful reminder of the immense scale of the cosmos. However, the distances to stars in our galaxy are tiny in comparison to distances to other galaxies.\\  
Since antiquity, observers have noted the existence of nebulous stars; diffuse smudgy or cloudy looking stars. Some of them turned out to be what we now know as nebulae, the places where stars form. Many turned out to be something else entirely. It wasn't until the 1920s when it was confirmed that many of these nebulous stars were in fact completely different galaxies, whole other sets of billions of stars like the Milky Way, far beyond our own. \\ 

We now know the Milky Way is but one of the billions of galaxies in the universe. Looking back at how astronomy developed this concept over time one can see how philosophers and scientists struggled with comprehending the nature of galaxies, and thus the enormity of our universe.
\section{The Milky Way Resolves into More Stars}
To the naked eye it is unclear exactly what the Milky Way is. In ancient Greece, the atomist philosopher Democritus had proposed that the bright band of light might consist of distant stars. The atomists' views were eclipsed by Aristotle's perspectives on the universe.\\ 

In Aristotelian Cosmology, the Milky Way was understood to be the point where the celestial spheres came into contact with the terrestrial spheres. One of the important observations Galileo noted in his 1610 Sidereus Nuncius was that, under the view of a telescope, parts of the Milky Way resolved into a cluster of many stars. Once again a weakness in Aristotelian Cosmology was found - the Milky Way wasn't the result of interactions between the terrestrial and celestial spheres. Galileo's observations demonstrated the Milky Way was a massive grouping of individual stars, planets and other nebulous elements.

\section{Island Universes and External Creations}
In 1750, English astronomer Thomas Wright, published An original theory or new hypothesis of the Universe.  In this book, Wright speculated that the Milky Way was a flat layer of stars, a part of which which was our solar system.\\

Beyond this he suggested that many of the very faint nebulae "in all likelihood may be external creation, bordering upon the known one, too remote for even our telescopes to reach." The idea that the faint nebulae could be their own "external creations" suggested the universe was much large than previously imagined. In 1755, philosopher Immanuel Kant elaborated on Wright's ideas and referred to these faint nebulae as "island universes." Both the notions of external creations and island universes struggled to capture the implications of this new larger scale of the universe. Beyond the fact that our sun was a star, could nebulae be their own universes or completely separate creations?

\section{Surveying the Milky Way}
In the 1780s William Herschel surveyed the stars in a range of different directions. He found that the stars were much denser on one side of the sky than those of the other side.\\

His son John Herschel conducted a similar study of the sky in the southern hemisphere and found the same pattern. What they were seeing was the core of the Milky Way galaxy, where there is a much greater density of stars.\\

Herschel had placed our sun nearly at the center of the Milky Way; it wouldn't be until the 1920's when Harlow Shapley's demonstrated that our sun was far from the center of the Milky Way.\\

\section{Andromeda and Other Nebulae}
Nebulous stars have been observed for thousands of years. In 964 Islamic astronomer Al-Sufi had observed and recorded what he called "a small cloud" in an illustration of the constellation Andromeda. We now understand this description as the Andromeda galaxy. Only with the advent and refinement of the telescope was it possible to start to document different kinds of nebulous stars.\\

As already mentioned, Thomas Wright and Immanuel Kant had published their speculations that faint nebulous stars like this were in fact independent entities like the Milky Way. In the late 18th century Charles Messier compiled a catalog of the 109 brightest nebulae, which was followed by a William Herschel's much larger catalog of over 5,000. Even while documenting all of these nebulae it remained unclear as to exactly what they were.
\section{Finding and Interpreting Red Shift}
Studying the light spectrum of nebulae like Andromeda would ultimately provide the information about what exactly these objects were. A range of astronomers worked on this issue in the early 20th century. In 1912 astronomer Vesto Slipher studied the light spectra of some of the brightest nebulae. He was interested in determining if they were made of the kinds of chemicals one would expect to find in a planetary system.

Slipher found something very interesting - it is possible to calculate the relative speed and distance of a star or nebulae is moving by examining the light spectrum it gives off and seeing how much the indicators for elements have shifted into the blue or red color spectrum. Objects shifted blue are moving closer to us and red shifted objects are moving away from us. In Slipher's analysis, the spectrums for the nebula were shifted so far into the red that these nebulae must be moving away from the earth at speeds beyond the escape velocity of the Milky Way. Along with this evidence, in 1917 Herber Curtis observed a nova, the brightening of an exploding star, inside the Andromeda Nebula. Looking back over photographs of the Nebula he was able to document 11 more novae that were on average 10 times fainter than those of the Milky Way. The evidence was mounting to suggest that these nebulae were well outside the Milky Way.\\

In 1920, Harlow Shapley and Heber Curtis debated the nature of the Milky Way, nebulae and the scale of the universe. Using the 100 inch telescope at Mt. Wilson, Edwin Hubble was able to resolve the edges of some spiral nebulae to identify they were in fact collections of stars, some of which matched standard patterns that enable astronomers to calculate that the stars were too distant to be part of the Milky Way.  Thus, the idea of the Milky Way as just one of many galaxies came to be the dominant scientific perspective.\\

Where the Earth was once understood to be the center of a relatively small universe we have come to understand it as one world orbiting one of the 300 billion stars in our galaxy which is itself just one of more than a hundred billion of galaxies in the observable universe. Even today it remains difficult to grasp just how tiny and small our planet is in the vastness of the observable universe.

\end{document}